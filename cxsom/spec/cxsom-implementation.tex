\section{Implementation}

\subsection{Variables}

Variables are stored in \Code{.var} files. They are binary files. When unsigned integers are mentionned in the following, they are 8-byte unsigned values, starting from most significant bytes first (big endian). The file is organized as follows, in that order:
\begin{itemize}
\item 64 bytes: They contain an ascii version of the type, ended by \Code{'\textbackslash n'}, and complemented with 0 bytes (padding) until the whole description's length is exactly 64-bytes.
\item 8 bytes: An unsigned integer representing the cache size. This is used by the simulator to determine the size of the cache to be associated to that variable during the simulation.
\item 8 bytes: An unsigned integer representing the buffer size $s$. Indeed, the file stores an history of the variable, i.e. the values from $\DI\DftTL\DftVar 0$ to $\DftDI$. Although this theoretically represents $\DftInst+1$ values, all of them may not be present in the file. The file is rather a circular buffer, with a limitted size $s$. So only values from $\DI\DftTL\DftVar {\DftInst - s + 1}$ to $\DftDI$ are stored. Once determined in the file, the value of the buffer size cannot be changed.
\item 8 bytes: The highest time in the file. This is the value $\DftInst$ mentionned above. If the file is empty (i.e. even the the first $\DI\DftTL\DftVar 0$ is not stored yet), the 8~bytes are \Code{0xFFFFFFFFFFFFFFFF}.
\item 8 bytes: The next free position in the file. The file contain a range of $s$ values, but it is a circular buffer. This bytes tells which index (starting from zero) is the next free position (i.e. where the $\DI\DftTL\DftVar{\DftInst+1}$ has to be stored.
\item from 0 to at most $s\times\Group{d+1}$ bytes: This is the data. Each datum is a $\DI\DftTL\DftVar\DftIInst$ value, {\em preceeded} by a boolean byte. So if $d$ is the number of bytes required for storing a datum, a slot requires $d+1$ bytes. The file buffer contains then from 0 to at most $s$ slots for storing values. If the boolean byte is 0 in a slot, the datum in that slot is considered as undetermined yet (i.e. $\Busy$). Otherwise, the $d$ bytes following the boolean byte describe the datum value stored in this slot.
\end{itemize}
